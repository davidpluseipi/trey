\documentclass{article}
\usepackage{graphics}

\begin{document}
\title{Project 3}
\author{D. M.}
\date{14 September 2020}
\maketitle

\section{Introduction}
\label{intro}
The purpose of this project is to learn how to find files in a particular folder.

\section{Command}
\label{sec:1}
The command I ended up using for this was
\begin{verbatim}
find . -name "oski*.*" -print "%f\n" > results.res
\end{verbatim}

\subsection{Find}
\label{sec:2}
The command \verb|find| is followed by a folder path.

\subsection{Options}
\label{sec:3}

The two options used were \verb|-name| and \verb|-printf|. The option
\verb|-name| requires, as an input argument, the name of the file
you're looking for in double quotes. Here I used * (wild card) to tell
it I wanted to find a file with any extension whose filename started with
\verb|oski|. The option \verb|-printf| was used with the argument \verb|%f\n|
to tell it to print out 1) just the filename (\verb|%f|), and 2) print
each file's name on a different line (\verb|\n|). The rest of the command
\verb|> results.res| just places those results in the .res file for storage
and/or later use.

\section{Process}
\label{sec:4}
To complete these tasks and best determine which functions and options to use,
I would repeat the following process.
\begin{enumerate}
	\item Review the command's help with \verb|command --help|
	\item Review the command's doc on gnu.org.
	\item Do a Google search within the gnu.org domain searching for
 '\verb|gnu.org: search pattern|'
	\item Do a Google search within the stackoverflow.com domain
	\item Search Google, letting Google find the best results
	\item Contact someone else for help or advice
\end{enumerate}

\begin{thebibliography}{9}
\bibitem{RefA}
MacKenzie, D., Youngman, J. (2019). \textit{Finding Files}. Free Software Foundation. 
https://www.GNU.org/software/findutils/manual/find.pdf
\end{thebibliography}
\end{document}
